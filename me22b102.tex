\section*{me22b102}
My favorite equation is \footnote[1]{Becerril, R., Guzmán, F. S., Rendón-Romero, A., and Valdez-Alvarado, S. (2008). Solving the time-dependent schrödinger equation using finite difference methods. Revista Mexicana De Fisica E, 54(2), 120-132. Retrieved from www.scopus.com } \textbf{general form of the one-dimensional  Schr$\mathbf{\ddot{o}}$dinger equation  Equation} which is as follows: 
\begin{eqnarray}
    \label{Schrodinger's equation}
    i\hbar{\partial \Psi \over \partial t} = {- \hbar^2 \over 2m} {\partial^2 \Psi(x,t) \over \partial x^2} + V(x,t)\Psi(x,t)
    \\ \textrm{where, } \hbar = \frac{h}{2 \pi}    
\end{eqnarray}
The time-dependent Schrödinger equation, is used to find the time dependence of the wavefunction.
Schrödinger Equation is a mathematical expression which describes the change of a physical quantity
over time in which the quantum effects like wave-particle duality are significant. The Schrödinger
Equation has two forms: the time-dependent Schrödinger Equation and the time-independent Schrödinger
Equation. The time-dependent Schrödinger Wave Equation derivation is provided here. The left hand side 
of the equation is just constant multiplied by the rate of change of wavefunction and the right hand side 
of the equation is just the Hamiltonian of the system multiplied with the wavefunction. Written by \textbf{Akshay Govind S 
GitHub id : wrb2222}

\section*{Details}
\begin{itemize}
    \item Name: Akshay Govind S 
    \item Roll Number: me22b102
    \item GitHub ID : wrb2222
\end{itemize}



