\documentclass{article}
\usepackage[utf8]{inputenc}
\usepackage{url}

\title{ID2090 Assignment 3}
\author{Rohitashwa Sahu me22b011}
\date{ 3rd February 2023}

\begin{document}
\maketitle 

\section{ME22B011}
\subsection{Law of Mass Action}
The law of mass action is universal, applicable under any circumstance. However, for reactions that are complete, the result may not be very useful. We introduce the mass action law by using a general chemical reaction equation in which reactants A and B react to give products C and D.
\begin{equation}
    aA + bB \rightarrow cC + dD
\end{equation}
where a, b, c, d are the coefficients for a balanced chemical equation. The mass action law states that if the system is at equilibrium at a given temperature, then the following ratio is a constant:
\begin{equation}
    K_{eq} = \frac{[C]^c[D]^d}{[A]^a[B]^b}
\end{equation}
But the above equation is valid if the reaction is in equilibrium i.e 
\begin{equation}
    Q = K_{eq}
\end{equation}
If the system is NOT at equilibrium, the ratio is different from the equilibrium constant. In such cases, the ratio is called a reaction quotient which is designated as Q.
\begin{equation}
 Q = \frac{[C]^c[D]^d}{[A]^a[B]^b}   
\end{equation}
A system not at equilibrium tends to become at equilibrium, and the changes will cause changes in Q so that its value approaches the equilibrium constant, K:
\begin{equation}
    Q \rightarrow K_{eq}
\end{equation}
The mass action law gives us a general method to write the expression for the equilibrium constant of any reaction.
Thanks for Reading this Article




\footnote{The above information is hereby an excerpt from LibreTexts Chemistry}
\footnote{Url for the above article is "https://chem.libretexts.org/"}




\end{document}
