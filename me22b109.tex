\documentclass[a4paper,12pt]{article}
\begin{document}
\section{ME22B109}

\begin{equation}
    P = \sigma eAT^4
    \label{eqn:Stefan}
\end{equation}

In 1879, Joseph Stefan discovered experimentally the law according to which radiated power per unit surface of a black body is proportional to the fourth power of temperature\footnote{The above information has been written from Introduction to Quantum Mechanics 1:Thermal Radiation and Experimental Facts regarding Quantization of Matter authored by Ibrahima Sakho, the DOI of which is: 10.1002/9781119671459}. 
In the Law-\ref{eqn:Stefan} $ \sigma $ designates Stefan constant.In 1884, Ludwig Boltzmann, a PhD student under Stefan’s  supervision, provided the theoretical proof for the empirical law.  This is why it is often referred to as Stefan–Boltzmann Law.
A predominant application of Stefan-Boltzmann Law is to measure radius of stars.To calculate the radius of a star, its luminosity is taken into consideration. The luminosity is the total power discharged by the star in space. It depends on two factors, i.e., the temperature and surface area. The relationship between the temperature of an object, the surface area of the body, and the rate of radiation discharge is given by the Stephan-Boltzmann law. Hence, it can be used to calculate the radius of a star.
Thus,this paragraph gives a brief outlook into Stefan-Boltzmann Law.

Author:Aryan Muralidharan(ME22B109)

Github ID:AryanMuralidharan
\end{document}
