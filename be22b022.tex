\section{BE22B022}

Name : Jerry Jose\\
Github User-ID : n0xet3rnal

\begin{equation}
{\frac {\partial V}{\partial t}}+{\frac {1}{2}}\sigma ^{2}S^{2}{\frac {\partial ^{2}V}{\partial S^{2}}}+rS{\frac {\partial V}{\partial S}}-rV=0
\label{bs_pde}
\end{equation}
The Black-Scholes equation is a partial differential equation regarding the price evolution of a European call or option under the Black-Scholes model.
In the equation displayed above,
\begin{itemize}
\item V is the price of the option as a function of stock price S and time t
\item r is the risk-free interest rate
\item $\sigma$ is the volatility of the stock
\end{itemize}
This equation is built on the financial understanding that in a frictionless market(one without transaction costs),it is possible to hedge the option by buying and selling the underlying asset in a particular fashion,thereby eliminating risk.This implies the existence of a unique optimum price for the option,which is given by the Black-Scholes model.The boundary and final conditions are the points of difference between American and European options,as well as put,call
and other types of options.


\footnote{This equation was originally introduced in the Journal of Political Economy in a 1973 paper titled "The Pricing of Options and Corporate Liabilities" by Fischer Black and Myron S. Scholes.}
\footnote{A 2012 paper titled "Study of Black-Scholes model and its applications" in the journal Procedia Engineering, by A.S. Shinde and K.C. Takale was also referred.}
