\documentclass{article}
\usepackage{amsmath}
\begin{document}
\section*{BS22B028}
Equation of Quantize energy
$$ E = nhv $$
The equation E=nhv is used to find the energy of the photon. Energy can also be released or absorbed in small discrete packages that mean energy is quantized. Each
package is termed ”quantum”. Einstein expanded on this theory with Planck’s to show that radiant energy is also quantized and the quantum of radiant energy is called
a photon. So, Einstein shows that light behaves not only as a wave but also as a particle or photon and each photon of radiant has energy is directly related to the
frequency of that radiant determined by planks:
$$ E=nhv $$
where,\\
E=Energy of photon\\
h=Plank’s constant\\
V=frequency\\
n=no. of photons\\
It is important to note here that n is the actual number of photons (not the number of moles) \footnote{https://www.uochemists.com/enhv/}\\
Name:Sreelatha\\
Github user-id:Sreelathal
\end{document}
