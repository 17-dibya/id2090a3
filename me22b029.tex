%\documentclass[12pt,a4paper]{article}

%\begin{document}
\section{me22b029}

\vspace{10pt}

\subsection{The Euler-Lagrange Equation}
\vspace{20pt}

\begin{equation}
\label{eqn}
\frac{\textsl{d}}{\textsl{d} t}\left(\frac{\partial L}{\partial \dot{q}_i}\right) = \frac{\partial L}{\partial q_i}
\end{equation}

where,

L is called the \textit{Lagrangian},

$q_i$ is any generalized coordinate,

$\dot{q}_i$ is the time derivative of the generalized coordinate.


\subsection{About The Equation}
\vspace{10pt}

The Lagrangian \textbf{L} is defined as:
\vspace{10pt}
$L = T - V$

where,

T is the generalized Kinetic Energy (a function of $q_i$)

V is the generalized Potential (a function of $\dot{q}_i$)
\vspace{20pt}

The Euler-Lagrnage equation was developed in 1788 which is the basis of Lagrangian Mechanics.
The equation is an equivalent for Newton's Second Law and hence is a simple method to describe the model for mechanical systems.
Model problems involving Atwood machines, Harmonic oscillators, Charge in electromagnetic fields, Kepler problems of Earth in orbit of the Sun, Pendulum,
Molecular and fluid dynamics, etc. can solved by applying the Euler-Lagrange equation.  \\[\baselineskip]Reference
\footnote{Kasap, Zeki. (2014). EULER-LAGRANGE EQUATIONS ON THREE-DIMENSIONAL SPACE. Celal Bayar Üniversitesi Fen Bilimleri Dergisi. 10. 10.18466/cbufbe.21908.}

\vspace{90pt}
\footnotesize{Name: S. Pon Adithi}

\footnotesize{GitHub Username: adithi-velan}

%\end{document}
