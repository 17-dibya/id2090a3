\documentclass[12pt, a4paper]{article}
\include{amsmath}
\include{asmsymb}

\author{Aldis}
\title{Length contraction in relativistic speeds}
 
\date{Feb 04, 2023}

\begin{document}
\maketitle
\section{me22b070}

Length contraction\footnote{https://en.wikipedia.org/wiki/Length\_contraction}
is the phenomenon that a moving object's length is measured to be shorter than its proper length, which is the length as measured in the object's own rest frame.Length contraction was postulated by George FitzGerald and Hendrik Antoon Lorentz. Einstein showd that length contraction could be explained using special relativity, which changed the notions of space, time, and simultaneity.

\begin{equation}
	L=L_0(1-v^2/u^2)^\frac{1}{2}
\end{equation}

where 
\begin{itemize}
	\item L is the length observed by an observer in motion relative to the object
	\item $L_0$ is the proper length (the length of the object in its rest frame)
\end{itemize}

\begin{flushleft}
Name: Aldis Daniel G\\
Github User ID: catdisk04
\end{flushleft}

\end{document}
