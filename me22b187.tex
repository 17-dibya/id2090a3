%\documentclass[a4paper,12pt]{article}
%\usepackage[utf8]{inputenc}
%\usepackage{amsmath}
%\usepackage{amsfonts}
%\usepackage{amssymb}
%\usepackage{hyperref}
%\usepackage[left=2cm,right=2cm,top=1cm,bottom=2cm]{geometry}
%\author{me22b187}
%\title{My Favourite Equation}
%\begin{document}
%\maketitle
\section{me22b187}
\subsection{The Equation}

\begin{equation}
\label{rohan}
%\text{E} = \frac{\Text{P}_\text{o}}{\rho} + \frac{\text{v}^{2}_\text{o}}{2} + \text{gh}
	E=\frac{P_o}{\rho} +\frac{v^2_o}{2} + gh
\end{equation}
REFERENCE\footnote[1]{https://en.wikipedia.org/wiki/Bernoulli\%27s\_principle\#Notes}
(please copy of paste the url,There is a BUG-footnote in final pdf doesnot recognise link after a \_ )

where:
\begin{enumerate}
	\item	E is a constant,
	\item	$ P_o $ is the pressure at a point,
	\item   $\rho$ is the density of the fluids at all points,
	\item	$ v_o $ is the fluid flow speed at a point,
	\item	$ g $ is the acceleration due to gravity,
	\item	$ h $ is the elevation of the point above a reference plane with cartesian system of coordinates
%	$$\text{{P}_\text{o} is the pressure at a point,} $$
%	$$\text{{\pho} is the density of fluid at all points,} $$
%	$$\text{{v}_\text{o} is the fluid flow speed at a point,} $$
%	$$\text{g is the acceleration due to gravity} $$
%	$$\text{h is the elevation of the point above a reference plane with cartesian system of coordinates} $$
\end{enumerate}
\subsection{Description and Theory}
This Equation is called the Bernoulli's principle.It is for the flow of imcompressible fluids only and is derived from conservation of energy.
Before that there are some assumptions made
\subsubsection{Assumptions}
\begin{enumerate}
	\item The flow must be steady, that is, the flow parameters (velocity, density, etc.) at any point cannot change with time,
	\item The flow must be incompressible as said before even though pressure varies, the density must remain constant along a streamline
        \item Friction by viscous forces must be negligible.
\end{enumerate}
Here the E can be generalised as Total energy.This states that ina steady flow the sum of all foems of energy in a fluid is same at all points that are free of viscous fores.Thats is sum of kinetic,potential and internal energy remains constant.so when travelling horizontally the
the speed is maximum where the pressure is minimun and vise versa.



\subsection{Name and Github id}
Rohan B \\ 
rohanbaraniraj@outlook.com






%\end{document}
