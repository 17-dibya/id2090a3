%\documentclass{article}
%\begin{document}
\section{ME22B018}
\begin{center}
	{\large TAYLOR'S THEOREM}
\end{center}
Let n \( \epsilon \) N. Suppose that \( f^{n}(x) \) is continuous on [a,b] and is differentiable on (a,b). Then there exists a point c \( \epsilon \) (a,b) such that

\[ f(a) = f(x) + f'(x)(a-x) + \cdots + f^{n-1}(x)\frac{(a-x)^{n-1}}{(n-1)\mathrm{!}} + f^{n}(\epsilon)\frac{(a-x)^{n}}{n\mathrm{!}} \]\\In calculus, Taylor's theorem gives an approximation of a n-times differentiable function about a given point by a polynomial of degree n, called the nth order Taylor polynomial.\\[\baselineskip]
The first-order Taylor polynomial is the linear approximation of the function, and the second-order Taylor polynomial is often referred to as the quadratic approximation. Taylor's theorem was named after the mathematician Brook Taylor, who stated a version of it in 1715, although an earlier version of the result was already mentioned by James Gregory in 1671.\\[\baselineskip]
Name: Shrinidi Kuppurajan\\
GitHub user-id: shrinidik\\[\baselineskip]
Reference\footnote{šikić, Zvonimir. (1990). Taylor's theorem. International Journal of Mathematical Education in Science and Technology. 21. 111-115. 10.1080/0020739900210115.}
%\end{document}
