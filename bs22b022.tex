\documentclass{article}


\usepackage[english]{babel}

\usepackage[letterpaper,top=2cm,bottom=2cm,left=3cm,right=3cm,marginparwidth=1.75cm]{geometry}


\usepackage{amsmath}
\usepackage{graphicx}
\usepackage[colorlinks=true, allcolors=blue]{hyperref}

\title{Assignment 3}
\author{R Karthik}

\begin{document}
\maketitle



\section{BS22B022}
\vspace{2mm}
Name: R Karthik

\vspace{2mm}
GitHub ID: Karthik0042
\subsection{The Equation}
$R_{\mu \nu} - {1 \over 2}g_{\mu \nu}\,R + g_{\mu \nu} \Lambda = 
 {8 \pi G \over c^4} T_{\mu \nu}$\footnote{https://en.wikipedia.org/wiki/Einstein_field_equations}
 
\vspace{5mm} 
 $R_{\mu \nu}$ : Ricci Tensor
 
\vspace{2mm}
$g_{\mu \nu}$ : Metric Tensor

\vspace{2mm}
$T_{\mu \nu}$ : Stress Energy Tensor

\vspace{2mm}
$G$ : Gravitational Constant

\vspace{2mm}
$c$ : speed of light

\vspace{2mm}
\subsection{Field Equation}
Einstein's field equation is a set of mathematical relationships that describe how mass and energy curve spacetime. It is the central equation of Einstein's theory of general relativity, which explains the behavior of objects in a gravitational field. The equation states that the curvature of spacetime is proportional to the distribution of matter and energy in that space. This means that massive objects cause spacetime to curve around them, and lighter objects will follow that curved path, giving the appearance of gravity. The equation also has important implications for the evolution of the universe, including the prediction of black holes and the observed accelerating expansion of the universe.



\end{document}