\documentclass{article}
\usepackage[utf8]{inputenc}
\title{me22b095.tex}
\author{Aditya Mahapatra}
\date{ 4th February 2023}

\begin{document}

\maketitle

\section{ME22B095}
Every object in the universe attracts every
other object with a force which is proportional
to the product of their masses and inversely
proportional to the square of the distance
between them. The force is along the line
joining the centres of two objects.

 Let two objects A and B of masses M and
m lie at a distance d from each other as shown in the figure below.
Let the force of attraction between
two objects be $\vec{F}$. According to the universal
law of gravitation, the force between two
objects is directly proportional to the product
of their masses.


The Force equation thus takes the form:

\begin{equation}
    {\bf \vec F}\footnote{Referred from NCERT Class 10 Physics,Chapter 10:Gravitation ,Section 10.1 ; https://ncert.nic.in/textbook/pdf/iesc110.pdf}={GmM\vec{d}\over d^3} 
\end{equation}

where G is the constant of proportionality and
is called the universal gravitation constant
and $\vec{d}$ is the direction in which the force acts.

 In today's language, the law states that every point mass attracts every other point mass by a force acting along the line intersecting the two points. The force is proportional to the product of the two masses, and inversely proportional to the square of the distance between them.
 

\end{document}
