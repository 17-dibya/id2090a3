\documentclass[12pt,a4paper]{article}
\usepackage[T1]{fontenc}
\usepackage{mathtools}
\usepackage{physics}

\title{\underline{ID2090 Assignment 3(Nov-March)}}
\begin{document}
\date{3rd February 2023}
\maketitle
\section*{BE22B032}

Name: Saahas Vijayalakshmi Rajaram \\
Github ID: SaahasVR

\subsection*{\underline{Navier-Stokes Equations}}

    \footnotemark{In fluid dynamics, the Navier-Stokes equations are equations that describe the three-dimensional motion of viscous fluid substances. These equations are named after Claude-Louis Navier (1785-1836) and George Gabriel Stokes (1819-1903). In situations in which there are no strong temperature gradients in the fluid, these equations provide a very good approximation of reality. \\
    
    The Navier-Stokes equations consists of a time-dependent continuity equation for conservation of mass, three time-dependent conservation of momentum equations and a time-dependent conservation of energy equation. There are four independent variables in the problem, the x, y, and z spatial coordinates of some domain, and the time t.}

\subsection*{\underline{Momentum equation}}

\begin{equation}
    \nabla . \vec{V} = 0
    \label{eqn:momentum}
\end{equation}

\subsection*{\underline{Continuity equation}}

\begin{equation}
    {\rho [ \frac{\partial{V}}{\partial{t}}+(V.\nabla)V ] = -\nabla p + \rho \vec{g} + {\mu} {\nabla}^2 \vec{V}}
    \label{eqn:continuity}
\end{equation}


\footnotetext{https://www.thermal-engineering.org/what-is-navier-stokes-equation-definition/}
\end{document}










