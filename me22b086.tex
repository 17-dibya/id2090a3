\title{ID2090 Assignment 3}
\Author{Dibyajyoti Nayak ME22B086}
\date{8 february 2023}

\begin{document}
\maketitle

\section{ME22B086}

Name - Dibyajyoti Nayak \\
Roll No. : me22b086 \\
Github user ID - 17-dibya \\

\subsection{linear diophantine equation}
In mathematics, a Diophantine equation is an equation, typically a polynomial equation in two or more unknowns with integer coefficients, such that the only solutions of 
interest are the integer ones. A linear Diophantine equation equates to a constant the sum of two or more monomials, each of degree one.

\begin{equation}
   ax + by = c
\end{equation}
where a, b and c are given integers. The solutions are described by the following theorem:

This Diophantine equation has a solution (where x and y are integers) if and only if c is a multiple of the greatest common divisor of a and b. Moreover, if (x, y) is a 
solution, then the other solutions have the form (x + kv, y − ku), where k is an arbitrary integer, and u and v are the quotients of a and b (respectively) by the 
greatest common divisor of a and b.

\footnote{
\url{https://en.wikipedia.org/wiki/Diophantine_equation}
}
