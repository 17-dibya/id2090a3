%\documentclass[12pt]{article}
%\begin{document}
\section{ME22B014}
My name is  \textbf{\textit{Chinmaya Holla V}}.\newline
My GitHub User Id is \textbf{\textit{ChinmayaHolla}}.
\newline

The Equation which is done here is the Thomas-Fermi Equation
\begin{equation}
\frac{d^2y}{dx^2} = \frac{y^\frac{3}{2}}{x^\frac{1}{2}}
\end{equation}
\newline
\newline
Created independently by Llewellyn H. Thomas and Enrico Fermi around 1926, the Thomas-Fermi model is a quantum mechanical theory for the electronic structure of a many-bodysystem. This statistical model was developed separately from wave function theory by being formulated in terms of electron density. The idea of the model is that given a large atom, with many electrons, one can approximately model it by a simple nonlinear problem for a specified charge density. In a mathematical framework, one can take the qualitative and quantitative physical assumptions imposed by the Thomas-Fermi model and solve the model through the Direct Method of calculus of variations. In mathematics, the Thomas–Fermi equation for the neutral atom is a second order non-linear ordinary differential equation,which can be derived by applying the Thomas–Fermi model to atoms
\newline
\\
The Thomas-Fermi model is a relatively crude model of multi-electron atoms that is useful for many purposes in a first approximation. The basic idea is to represent the electron cloud surrounding the nucleus as a zero-temperature, negatively charged, degenerate Fermi-Dirac fluid, which is held in a condition of hydrostatic equilibrium by a balance between pressure gradients and electrostatic forces
\newline
In 1962, Edward Teller showed that Thomas–Fermi theory cannot describe molecular bonding – the energy of any molecule calculated with TF theory is higher than the sum of the energies of the constituent atoms. More generally, the total energy of a molecule decreases when the bond lengths are uniformly increased.This can be overcome by improving the expression for the kinetic energy.
\subsection{Dirac Correction}
Another scientist named Dirac modified the equation and gave another equation considering the exchange effects. This equation is called Thomas-Fermi-Dirac Equation.
\newline
\newline
\begin{equation}
\frac{d^2 \psi}{dx^2}=x(\epsilon+\frac{\psi^\frac{1}{2}}{x^\frac{1}{2}}  )^3
\end{equation}
\newline
However, the Thomas–Fermi–Dirac theory remained rather inaccurate for most applications. The largest source of error was in the representation of the kinetic energy, followed by the errors in the exchange energy, and due to the complete neglect of electron correlation.The notable historical improvement to the Thomas–Fermi kinetic energy is the Weizsäcker (1935) correction.
\subsection{Weizsäcker Correction}
The Thomas-Fermi and related theories are attractive because of their simplicity, are not satisfactory for atomic problems because they yield an electron density with incorrect behavior very close to and very far away from the nucleus. Von Weizsacker suggested the addition of an inhomogeneity correction to the kinetic energy density.
\begin{equation}
T_w=\frac{1}{8}\frac{h^2}{m}\int \frac{|\nabla n(r)|^2}{n(r)}d^3r
\end{equation}

\footnote{Equations of State of Elements Based on the Generalized Fermi-Thomas Theory,Feynman, R. P. and Metropolis, N. and Teller, E.,https://link.aps.org/doi/10.1103/PhysRev.75.1561}

%\footnote{Proximity forces,Błocki, J. and Randrup, J. and Świa̧tecki, W.J. and Tsang, C.F.,https://www.scopus.com/inward/record.uri?eid=2-s2.0-0001587411&doi=10.1016\%2f0003-4916\%2877\%2990249-4&partnerID=40\&md5=1836703cf6e2bdcf6cd273cabec1bc23}\newline

%\end{document}
